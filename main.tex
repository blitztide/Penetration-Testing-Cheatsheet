\documentclass[a4paper,10pt]{article}
\usepackage[utf8]{inputenc}
\usepackage[dvipsnames]{xcolor}
\usepackage[defaultsans]{droidsans}
\renewcommand*\familydefault{\sfdefault} %% Only if the base font of the document is to be typewriter style
\usepackage[T1]{fontenc}
\usepackage{lastpage}
%\usepackage{draftwatermark}
\usepackage{subfiles}
\usepackage{tabularx}
\usepackage{colortbl}
\usepackage{multicol}
\usepackage[toc]{appendix}
\usepackage{pdfpages}
\usepackage{float}
\usepackage{tocloft} % fix formatting on ToC
\tocloftpagestyle{fancy}
\usepackage{tcolorbox} %Pretty boxes
\tcbuselibrary{listings} %Allow pretty boxes to be compatible with listings
\tcbuselibrary{breakable} %allow pretty boxes to flow over page boundaries
\usepackage{minted} %Pretty code
\usepackage[linktocpage=true]{hyperref}
\usepackage{subcaption} % Allows subfigures
%Fix alignment on ToC
\cftsetindents{section}{0em}{1.5em}
\cftsetindents{subsection}{3em}{4em}
\cftsetindents{subsubsection}{6em}{4em}
\renewcommand{\arraystretch}{1.3}
\usepackage[top=2cm, bottom=2cm, outer=1cm, inner=1cm]{geometry}
\usepackage{wallpaper}
%\setlength{\columnseprule}{1pt}
\begin{document}
\ThisCenterWallPaper{1.2}{title.jpg}
\centering
\begin{tcolorbox}[width=0.75\linewidth]
	\huge
	\centering
	\textbf{Penetration Testing Cheat Sheet}
\end{tcolorbox}
\vfill
\begin{tcolorbox}[width=0.6\linewidth]
	\centering
	\large
	First Edition
\end{tcolorbox}
\clearpage
\Huge{\textbf{Network Enumeration}}
\newline
\normalsize
\begin{multicols}{2}
\begin{tcolorbox}[breakable, title=Nmap]
\textbf{Switches}:\\
\textit{-oA} output all formats\\
\textit{-O} Operating System enumeration\\
\textit{-p} Port specification (-p- for all ports)\\
\textit{-sC} Script Enumeration\\
\textit{-sV} Version Enumeration (full tcp connect)\\
\textit{-sU} UDP enumeration\\
\textit{--script=} Script selection\\
\textit{-Pn} Disable ping probes\\
\textit{-iL} Include hosts from file\\
\textit{--script-args} Provide arguments to NMAP scripts\\
\newline
\textbf{Useful Scripts}:\\
\textit{default} - Default Scripts\\
\textit{vuln} - Enumerate vulnerabilities\\
\textit{ftp-*} - All FTP Scripts\\
\textit{http-*} - All HTTP Scripts\\
\textit{smb-*} - All SMB Scripts\\
\textit{nfs-*} - All NFS Scripts\\
\textit{ldap-search} - Performs ldap search\\
\textit{vulners} Searches for CVEs on returned services\\
\end{tcolorbox}
\begin{tcolorbox}[breakable, title=Windows commands]
\textit{arp -a} Show ARP table\\
\textit{ipconfig /all} Shows IP configuration\\
\textit{ping} ICMP Echo requests\\
\end{tcolorbox}
\begin{tcolorbox}[breakable, title=Linux commands]
\textit{arp -a} Show ARP table\\
\textit{ping} ICMP Echo requests\\
\textit{ip address} Show IP configuration\\
\textit{ifconfig} Show interface configuration\\
\end{tcolorbox}
\begin{tcolorbox}[breakable,title=Packet Capture]
\begin{itemize}
	\itemsep0em
	\item \textbf{wireshark}
	\item \textbf{tshark}
	\item \textbf{tcpdump}
	\item \textbf{netsh trace start capture=yes; netsh trace stop}
\end{itemize}
\end{tcolorbox}
\end{multicols}

\vspace{1cm}
\Huge{\textbf{Service Enumeration}}
\newline
\normalsize
\begin{multicols}{2}
\begin{tcolorbox}[breakable, title=FTP]
Checklist:
\begin{itemize}
	\item Anonymous Access
	\item Vulnerabilities\\ Service name be taken from banner grabs
	\item Default Credentials
\end{itemize}
Tools:
\begin{itemize}
	\item Filezilla
	\item ftp (inbuilt linux)
\end{itemize}
\end{tcolorbox}
\begin{tcolorbox}[breakable, title=DNS]
Checklist:
\begin{itemize}
	\itemsep0em 
	\item look for PTR records for the DNS server IP address
	\item look for PTR records for whole IP range
	\item Check for zone transfers
	\item Enumerate Windows AD DNS entries
	\item Enumerate MX records
	\item Enumerate SPF records (Email Security)
	\item Enumerate DKIM records (Email Security)
	\item Enumerate DMARC records
\end{itemize}
Tools:
\begin{itemize}
	\itemsep0em 
	\item \textbf{dig}
	\item \textbf{host}
	\item \textbf{nslookup}
	\item \textbf{gobuster}
\end{itemize}
\end{tcolorbox}
\begin{tcolorbox}[breakable,title=Group Policy]
Checklist:
\begin{itemize}
	\itemsep0em
	\item Stored Credentials
	\item Automated Scripts (privesc)
\end{itemize}
Tools:
\begin{itemize}
	\itemsep0em
	\item \textbf{Active Directory Group Policy Manager}
	\item \textbf{SYSVOL}
\end{itemize}
\end{tcolorbox}
\begin{tcolorbox}[title=HTTP/HTTPS]
Checklist:
\begin{itemize}
	\itemsep0em 
	\item Check common files (robots.txt .htaccess)
	\item Identify Web technologies:
	\item HTTP Server (IIS, Apache, Nginx)
	\item Preprocessors (Ruby on Rails, PHP, ASP)
	\item Web Applications (Wordpress, Drupal, Sharepoint)
	\item SSL Certificate Enumeration (Domain names, locations)
	\item Web Application Firewalls
	\item Check for VHOSTS (DNS enumeration can help)
	\item Check authentication technologies (JWT, Cookies, SAML, SSO, AD)
	\item Check for XSS
	\item Check for SQLi
	\item Check for LFI/RFI
	\item Check for shellshock
\end{itemize}
Tools:
\begin{itemize}
	\itemsep0em	
	\item \textbf{Gobuster} DNS/Directory brute forcer
	\item \textbf{dirb} Directory brute forcer
	\item \textbf{Burp Suite} HTTP Proxy
	\item \textbf{Zed Attack Proxy} HTTP Proxy
	\item \textbf{wpscan} WordPress vulnerability scanner
	\item \textbf{drupalgeddon} Drupal vulnerability scanner
	\item \textbf{curl} HTTP request tool
	\item \textbf{wget} HTTP downloader
	\item \textbf{sqlmap} SQL Injection scanner
	\item \textbf{nmap --script=http-shellshock}
\end{itemize}
\end{tcolorbox}

\begin{tcolorbox}[breakable, title=KERBEROS]
Checklist:
\begin{itemize}
	\itemsep0em
	\item Enumerate Domain Users
	\item Pass the Ticket
	\item ASREPRoast
	\item Pass the Key
	\item Silver Ticket
	\item Golden Ticket
\end{itemize}
Tools:
\begin{itemize}
	\itemsep0em
	\item \textbf{Mimikatz}
	\item \textbf{Impacket}
	\item \textbf{PsExec}
	\item \textbf{kerbrute} Kerberos password brute forcing
	\item \textbf{GetNPUsers.py} Locate ASREPRoastable users
\end{itemize}
\end{tcolorbox}
\begin{tcolorbox}[breakable, title=LDAP]
Checklist:
\begin{itemize}
	\itemsep0em
	\item Users and Groups
	\item Privileged Groups
	\item Company layout
	\item Computers
\end{itemize}
Tools:
\begin{itemize}
	\itemsep0em
	\item \textbf{Active Directory Users and Computers}
	\item \textbf{python3 ldap3 module}
	\item \textbf{powershell}
\end{itemize}
\end{tcolorbox}
\begin{tcolorbox}[breakable, title=MYSQL/MariaDB]
Checklist:
\begin{itemize}
	\itemsep0em
	\item Check for CVEs
	\item Enumerate interesting databases
\end{itemize}
Tools:
\begin{itemize}
	\itemsep0em
	\item \textbf{mysql}
\end{itemize}
\end{tcolorbox}
\begin{tcolorbox}[breakable,title=NETBIOS]
Checklist:
\begin{itemize}
	\itemsep0em
	\item Discover Hosts on network segment
\end{itemize}
Tools:
\begin{itemize}
	\itemsep0em
	\item \textbf{nbtscan}
\end{itemize}
\end{tcolorbox}
\begin{tcolorbox}[breakable, title=NFS]
Checklist:
\begin{itemize}
	\itemsep0em
	\item Anonymous/Guest access
	\item Sensitive file disclosure
\end{itemize}
Tools:
\begin{itemize}
	\itemsep0em
	\item \textbf{showmount}
	\item \textbf{nmap --script=nfs*}
\end{itemize}
\end{tcolorbox}
\begin{tcolorbox}[breakable,title=NTP]
Checklist:
\begin{itemize}
	\itemsep0em
	\item Synchronise time with host
	\item Enumerate connected clients
	\item Enumerate version information
\end{itemize}
Tools:
\begin{itemize}
	\itemsep0em
	\item \textbf{nmap --script=ntp-info}
	\item \textbf{ntpq}
	\item \textbf{ntpdc}
\end{itemize}
\end{tcolorbox}
\begin{tcolorbox}[breakable, title=RDP]
Checklist:
\begin{itemize}
	\itemsep0em
	\item Screenshots
	\item User Enumeration
	\item Check for RDP Vuln
	\item OS Version check
\end{itemize}
Tools:
\begin{itemize}
	\itemsep0em
	\item \textbf{mstsc.exe}
	\item \textbf{rdesktop}
	\item \textbf{remmina}
	\item \textbf{nmap}
	\item \textbf{crowbar}
\end{itemize}
\end{tcolorbox}
\begin{tcolorbox}[breakable, title=RPC]
Checklist:
\begin{itemize}
	\itemsep0em
	\item Anonymous login
	\item Enumerate Domain information
	\item Enumerate System information
\end{itemize}
Tools:
\begin{itemize}
	\itemsep0em
	\item \textbf{rpcclient}
\end{itemize}
\end{tcolorbox}
\begin{tcolorbox}[breakable, title=SMB]
Checklist:
\begin{itemize}
	\itemsep0em
	\item Anonymous/null session access
	\item Sensitive file disclosure
	\item Exposed shares
	\item SMB1 Vulnerability
\end{itemize}
Tools:
\begin{itemize}
	\itemsep0em
	\item \textbf{smbclient}
	\item \textbf{smbenum}
	\item \textbf{smbmap}
	\item \textbf{crackmapexec.py}
	\item \textbf{enum4linux}
\end{itemize}
\end{tcolorbox}
\begin{tcolorbox}[breakable, title=SMTP]
Checklist:
\begin{itemize}
	\itemsep0em
	\item Verify email address existence
	\item Send malicious emails
\end{itemize}
Tools:
\begin{itemize}
	\itemsep0em
	\item \textbf{nc}
\end{itemize}
\end{tcolorbox}
\begin{tcolorbox}[breakable, title=SNMP]
Checklist:
\begin{itemize}
	\itemsep0em
	\item Default Credentials
	\item Network configuration
\end{itemize}
Tools:
\begin{itemize}
	\itemsep0em
	\item \textbf{snmpwalk}
	\item \textbf{snmpenum}
	\item \textbf{snmpcheck}
\end{itemize}
\end{tcolorbox}
\begin{tcolorbox}[breakable,title=SSH]
Checklist:
\begin{itemize}
	\itemsep0em
	\item Reused SSH Keys %need to find the github repo
	\item Version Information
	\item Host information
\end{itemize}
Tools:
\begin{itemize}
	\itemsep0em
	\item \textbf{SSH}
	\item \textbf{nmap --script=ssh2-enum-algos}
\end{itemize}
\end{tcolorbox}
\begin{tcolorbox}[breakable, title=VNC]
Checklist:
\begin{itemize}
	\itemsep0em
	\item Enumerate running OS
	\item Screenshots
	\item Check CVEs
\end{itemize}
Tools:
\begin{itemize}
	\itemsep0em
	\item \textbf{vncviewer}
	\item \textbf{remmina}
	\item \textbf{nmap --script=vnc-info}
\end{itemize}
\end{tcolorbox}
\begin{tcolorbox}[breakable,title=WinRM]
Checklist:
\begin{itemize}
	\itemsep0em
	\item WMI Queries
	\item Pass the Hash
\end{itemize}
Tools:
\begin{itemize}
	\itemsep0em
	\item \textbf{winrs}
	\item \textbf{evil-winrm}
	\item \textbf{test-wsman}
\end{itemize}
\end{tcolorbox}
\end{multicols}
\newpage
\Huge{\textbf{Buffer Overflows}}
\newline
\normalsize
\begin{tcolorbox}
\textbf{Workflow}:
\begin{enumerate}
	\itemsep0em 
	\item Fuzz the application by sending larger buffers of valid Characters
	\item Once a buffer size has been confirmed, append full ASCII table to buffer
	\item Confirm bad characters in payload, remove them from the buffer and repeat step.
	\item Confirm the crash overwrites EIP and ESP with debugger running
	\item Use msf\_pattern\_create to create new buffer of correct size
	\item Confirm the crash with new patterned buffer and extract EIP
	\item use msf\_pattern\_find to find offset of EIP buffer
	\item use debugging tool to find a JMP ESP instruction
	\item use location of JMP ESP (Little Endian) to route execution to ESP
	\item place breakpoint on JMP ESP instruction and run exploit again
	\item Single step through the code to ensure that it is loading to the correct address (you may need to adjust the ESP)
	\item Create a malicious payload and insert into buffer (keeping buffer size the same) and ensuring that a sufficiently sized NOP sled has been included.
	\item Test execution on local copy
	\item Exploit
\end{enumerate}
\end{tcolorbox}
\begin{multicols}{2}
\begin{tcolorbox}[breakable,title=Windows Debugging Tools]
\textbf{Immunity Debugger}:\\
Immunity debugger contains mona.py which can be used to enumerate a vulnerable binary, it has the following commands:\\
\begin{itemize}
	\itemsep0em
	\item \textit{!mona modules}\\Lists all loaded modules
	\item\textit{!mona noaslr}\\Lists all loaded modules without ASLR
	\item\textit{!mona nosafesehasl}\\Lists all loaded modules without safe SEH or ASLR
	\item\textit{!mona nosafeseh}\\Lists all loaded modules without safe SEH
	\item\textit{!mona find -s ``$\backslash$xFF$\backslash$xFF'' -m ``module.dll''}\\ finds all addresses that match search term
\end{itemize}
\end{tcolorbox}
\begin{tcolorbox}[breakable,title=Linux Debugging Tools]
\begin{itemize}
	\itemsep0em 
	\item EDB\\Use the OpCodeSearcher plugin to find ESP - EIP ensuring that the memory region has execute permissions.
	\item GDB\\ You are on your own.
\end{itemize}
\end{tcolorbox}
\begin{tcolorbox}[breakable,title=Payload Generation]
\textbf{msfvenom Payloads}\\
List payloads with: msfvenom -l payloads
\begin{itemize}
	\itemsep0em
	\item shell/reverse\_tcp
	\item shell/bind\_tcp
\end{itemize}
\textbf{msfvenom formats}\\List formats with msfvenom -l formats
\begin{itemize}
	\itemsep0em
	\item \textbf{asp(x)}
	\item \textbf{dll}
	\item \textbf{elf(-so)}
	\item \textbf{exe(-service)}
	\item \textbf{hta-psh} hypertext application with embedded powershell
	\item \textbf{vbs}
	\item \textbf{c}
	\item \textbf{ps1}
	\item \textbf{python}
	\item \textbf{raw}
	\item \textbf{rb}
	\item \textbf{sh}
\end{itemize}
\end{tcolorbox}
\end{multicols}
\newpage
\Huge{\textbf{Windows Privesc}}
\newline
\normalsize
\begin{tcolorbox}[breakable,title=Generic]
\begin{itemize}
	\itemsep0em
	\item \textbf{Windows Exploit Suggester}
	\item \textbf{PSEXEC /s}
	\item Search for passwords in config files
	\item Search for passwords in Registry
	\item Search for sysprep.inf, sysprep.xml and Unattended.xml
	\item Group Policy Preference (Hardcoded encryption key)
	\item Install MSI as system user
	\item Check permissions for running services
	\item Check scheduled tasks
	\item Windows Version Specific Exploits
	\item Dump Password hashes
\end{itemize}
\end{tcolorbox}
\begin{multicols}{2}
\begin{tcolorbox}[breakable,title=JuicyPotato PrivEsc]
\textbf{Requirements}\\
Ran in the context of a Windows Service Account\\
\newline
\textbf{Affected Versions}
\begin{itemize}
	\itemsep0em
	\item Windows 7 Enterprise
	\item Windows 8.1 Enterprise
	\item Windows 10 Professional <1809
	\item Windows Server 2008 R2
	\item Windows Server 2012
	\item Windows Server 2012 R2
	\item Windows Server 2019
\end{itemize}
\end{tcolorbox}
\begin{tcolorbox}[breakable,title=RoguePotato PrivEsc]
	\textbf{Requirements}\\
	Ran in the context of a Windows Service Account\\
	\newline
	\textbf{Affected Versions}
	\begin{itemize}
		\itemsep0em
		\item Windows 10 Professional >=1809
		\item Windows Server 2019
	\end{itemize}
\end{tcolorbox}
\begin{tcolorbox}[breakable,title=Windows x86 'afd.sys' PrivEsc (MS11-046)]
\textbf{Requirements}\\
Local Shell on host\\
\textbf{Affected Versions}
\begin{itemize}
	\itemsep0em
	\item Windows XP Pro (x86)
	\item Windows Vista (x86)
	\item Windows 7 (x86)
\end{itemize}
\end{tcolorbox}
\begin{tcolorbox}[breakable,title=Windows XP]
\begin{itemize}
	\itemsep0em
	\item \textbf{upnphost service}\\Writeable by Authenticated users
\end{itemize}
\end{tcolorbox}
\begin{tcolorbox}[breakable,title=Windows Vista]
	
\end{tcolorbox}
\begin{tcolorbox}[breakable,title=Windows 7]
	
\end{tcolorbox}
\begin{tcolorbox}[breakable,title=Windows 8]
	
\end{tcolorbox}
\begin{tcolorbox}[breakable,title=Windows 8.1]
	
\end{tcolorbox}
\begin{tcolorbox}[breakable,title=Windows 10]
	
\end{tcolorbox}
\begin{tcolorbox}[breakable,title=Windows Server 2003]
	
\end{tcolorbox}
\begin{tcolorbox}[breakable,title=Windows Server 2008/R2]
	
\end{tcolorbox}
\begin{tcolorbox}[breakable,title=Windows Server 2012]
	
\end{tcolorbox}
\begin{tcolorbox}[breakable,title=Windows Server 2016]
	
\end{tcolorbox}
\begin{tcolorbox}[breakable,title=Windows Server 2019]
\end{tcolorbox}
\end{multicols}
\newpage
\Huge{\textbf{Linux Privesc}}
\newline
\normalsize
\begin{tcolorbox}[breakable,title=checklist]
\begin{itemize}
	\itemsep0em
	\item SUID Binaries
	\item Saved credentials
	\item Vulnerable scripts
	\item Service path vulnerabilities
	\item Sudo config
	\item Kernel Exploits
	\item SSH keys
	\item Crack shadow password hashes
\end{itemize}
\end{tcolorbox}
\begin{multicols}{2}
\begin{tcolorbox}[breakable,title=Scripts]
\begin{itemize}
	\itemsep0em
	\item \textbf{linenum.sh}
	\item \textbf{linuxprivchecker.py}
	\item \textbf{unix-privesc-check}
	\item \textbf{lynis}
\end{itemize}
\end{tcolorbox}
\begin{tcolorbox}[breakable,title=Shell escape sequences]
\textbf{Python}\\
import pty; pty.spawn("/bin/bash")\\
\textbf{Vim}\\
:set shell=/bin/sh\\
:shell\\
\textbf{nmap} - Versions 2.02 - 5.21 Inclusive\\
nmap --interactive\\
\textbf{awk}\\
awk 'BEGIN {system("/bin/sh")}'
\end{tcolorbox}
\end{multicols}
\newpage
\Huge{\textbf{Password Cracking}}
\newline
\normalsize
\begin{multicols}{2}
\begin{tcolorbox}[breakable,title=john]
\textbf{simple crack}\\
john --wordlist=wordlist.txt hashfile\\
\textbf{Zip file}\\
zip2john file.zip > file.john\\
john --format=zip file.john\\
\textbf{NTLMv2}\\
john --format=netntlmv2 hash.txt\\
\textbf{Kerberoast}\\
kirbi2john hashes.kirbi > kirbi.john\\
john --wordlist=password.lst kirbi.john\\
\textbf{Show john results}\\
john --show target.txt
\end{tcolorbox}
\begin{tcolorbox}[breakable,title=hashcat]
\textbf{Simple crack}\\ hashcat -m <hashmode> <hashfile> <wordlist>\\
\textbf{Common hashcat modes}
\vspace{-1em}
\begin{itemize}
	\itemsep0em
	\item \textbf{0} - MD5
	\item \textbf{100} - SHA1
	\item \textbf{1000} - NTLM
	\item \textbf{1100} - Domain Cached Credentials
	\item \textbf{1700} - SHA512
	\item \textbf{13100} - Kerberos TGS-REP etype 23
\end{itemize}
\textbf{Show hashcat results}\\hashcat --show <hashfile> -m <hash mode>
\end{tcolorbox}
\begin{tcolorbox}[breakable,title=Linux shadow hashes]
	\begin{itemize}
		\itemsep0em
		\item	\textbf{\$1} - MD5
		\item \textbf{\$2} - Blowfish
		\item \textbf{\$2a} - eksblowfish
		\item \textbf{\$5} - SHA-256
		\item \textbf{\$6} - SHA-512
	\end{itemize}
\textbf{Generating a Linux Shadow hash}\\
openssl passwd -1 -salt <salt> <password>
\end{tcolorbox}
\begin{tcolorbox}[breakable,title=online hash crackers]
\begin{itemize}
	\itemsep0em
	\item Crackstation
\end{itemize}
\end{tcolorbox}
\end{multicols}
\newpage
\Huge{\textbf{Kerberoasting}}
\newline
\normalsize
\vspace{1cm}
\begin{tcolorbox}[breakable,title=Notes]
Kerberoasting is a post-exploitation attack which extracts service account credential hashes for offline cracking.
Kerberoasting requires being within a domain context (Machine account, Domain User account).
\begin{enumerate}
	\itemsep0em
	\item Gain Domain Credentials
	\item Enumerate SPNs on network
	\item Request TGS for Service
	\item Dump ticket from memory
	\item Crack ticket offline to get plaintext credentials for service account
	\item Use credentials to:
	\subitem Generate fake tickets
	\subitem Log in as service account
\end{enumerate}
\end{tcolorbox}
\begin{tcolorbox}[breakable,title=Windows Tools and Scripts]
\begin{itemize}
	\itemsep0em
	\item Invoke-Kerberoast.ps1\footnote{https://github.com/EmpireProject/Empire/blob/master/data/module
		source/credentials/Invoke-Kerberoast.ps1}
	\item setSPN.exe
	\item Mimikatz.exe
\end{itemize}
\end{tcolorbox}
\begin{tcolorbox}[breakable,title=Linux Tools and Scripts]
\begin{itemize}
	\itemsep0em
	\item Impacket/GetNPUsers.py
	\item hashcat
	\item john
	\item tgsrepcrack.py
\end{itemize}
\end{tcolorbox}
\begin{tcolorbox}[title={Script to convert from Invoke-Kerberoast csv to Hashcat if not copy/pasted},breakable]
	\footnotesize
	\begin{minted}[linenos,numbersep=5pt,breaklines]{bash}
cat kerberoast.csv | iconv -f UTF-16 -t ASCII | cut -d ',' -f 2 | tail -n +3 | tr -d '\n'| sed -e 's/\"\"/\n/g'| sed -e 's/\"//g' > tickets.hashcat
	\end{minted}
\end{tcolorbox}
\newpage
\Huge{\textbf{Reference}}
\newline
\normalsize
\begin{tcolorbox}[breakable,title=IIS Versions]
\begin{itemize}
	\itemsep0em
	\item \textbf{IIS 1.0} Windows NT 3.51
	\item \textbf{IIS 2.0} Windows NT 4.0
	\item \textbf{IIS 3.0} Windows NT 4.0 SP 2
	\item \textbf{IIS 4.0} Windows NT 4.0 Option Pack
	\item \textbf{IIS 5.0} Windows 2000
	\item \textbf{IIS 5.1} Windows XP Professional
	\item \textbf{IIS 6.0} Windows Server 2003 / Windows XP Pro x64
	\item \textbf{IIS 7.0} Windows Server 2008 / Windows Vista
	\item \textbf{IIS 7.5} Windows Server 2008 R2 / Windows 7
	\item \textbf{IIS 8.0} Windows Server 2012 / Windows 8
	\item \textbf{IIS 8.5} Windows Server 2012 R2 / Windows 8.1
	\item \textbf{IIS 10.0} Windows Server 2016/ Windows 10 and onwards
\end{itemize}
\end{tcolorbox}
\begin{tcolorbox}[breakable,title=Creating Powershell encoded payloads]
	\begin{enumerate}
		\itemsep0em
		\item Convert to UTF-16LE
		\item base64 encode
	\end{enumerate}
\end{tcolorbox}
\begin{tcolorbox}[breakable,title=Common MAC Address vendors]
VMWare 00:50:56\\
VMWare 00:0C:29\\
VMWare 00:05:69\\
VMWare 00:1C:14\\
Virtualbox 08:00:27	\\
Virtualbox 52:54:00 \\
Virtualbox 00:21:F6
\end{tcolorbox}
\begin{tcolorbox}[breakable,title=Active Directory SRV records]
\begin{itemize}
	\itemsep0em
	\item \textbf{\_gc.\_tcp.<domain fqdn>} - Global Catalogue Server
	\item \textbf{\_ldap.\_tcp.<domain fqdn>} - LDAP server
	\item \textbf{\_kerberos.\_tcp.<domain fqdn>} - Kerberos Server (KDC)
	\item \textbf{\_kpasswd.\_tcp.<endpoint fqdn>} - Kerberos Server ()
\end{itemize}
\end{tcolorbox}
\begin{tcolorbox}[breakable,title=Wireshark Filters]
	Capture Filters:
	
	Packet Filters:
\end{tcolorbox}
\begin{tcolorbox}[breakable,title=tcpdump flags]
	
\end{tcolorbox}
\end{document}